% $Header: /cvsroot/latex-beamer/latex-beamer/doc/beamerug-installation.tex,v 1.11 2005/04/04 15:27:07 tantau Exp $

% Copyright 2003, 2004 by Till Tantau <tantau@users.sourceforge.net>.
%
% This program can be redistributed and/or modified under the terms
% of the GNU Public License, version 2.

\section{Installation}

\label{section-installation}


There are different ways of installing the \beamer\ class, depending
on your installation and needs. When installing the class, you may
need to install some other packages as well as described below. Before
installing, you may wish to review the \textsc{gpl} license under
which the class is distributed, see Section~\ref{section-license} below.



\subsection{Versions and Dependencies}

This documentation is part of version \version\ of the \beamer\
class. \beamer\ needs a reasonably recent version of several standard
packages to run and also the following versions of two special
packages (later versions should work, but not necessarily): 
\begin{itemize}
\item
  |pgf.sty| version \pgfversion,
\item
  |xcolor.sty| version \xcolorversion.
\end{itemize}

If you use |pdflatex| or |lyx|, which are optional, you need
\begin{itemize}
\item
  |lyx| version 1.3.3. Other versions might work.
\item
  |pdflatex| version 0.14 or higher. Earlier versions do not work. 
\end{itemize}



\subsection{Installation of Prebundled Packages}

I do not create or manage prebundled packages of \beamer, but,
fortunately, nice other people do. I cannot give detailed instructions
on how to install these packages, since I do not manage them, but I
\emph{can} tell you were to find them and I can tell you what these
nice people told me on how to install them. If you have a problem with
installing, you might wish to have a look at the Debian page or MikTeX
page first.


\subsubsection{Debian}

The command ``|aptitude install latex-beamer|'' should do the
trick. If necessary, the packages |pgf| and |latex-xcolor| will be
automatically installed. Sit back and relax. In detail, the following
packages are installed: 
\begin{verbatim}
http://packages.debian.org/latex-beamer
http://packages.debian.org/pgf
http://packages.debian.org/latex-xcolor
\end{verbatim}


\subsubsection{MiKTeX}

For MiK\TeX, use the update wizard to install the (latest versions of
the) packages called |latex-beamer|, |pgf|, and |xcolor|. 




\subsection{Installation in a texmf Tree}

If, for whatever reason, you do not wish to use a prebundled package,
the ``right'' way to install \beamer\ is to put it in a so-called
|texmf| tree. In the following, I explain how to do this.

Obtain the latest source version (ending |.tar.gz| or |.zip|) of the \beamer\
package from
\begin{verbatim}
http://sourceforge.net/projects/latex-beamer/
\end{verbatim}
(most likely, you have already done this). Next, you also need at
the \textsc{pgf} package, which can be found at the same
place. Finally, you need the  \textsc{xcolor} package, which can also
be found at that place (although the version on CTAN might be newer).

In all cases, the packages contain a bunch of files (for the \beamer\
class, |beamer.cls| is one of these files and happens to be the
most important one, for the \textsc{pgf} package |pgf.sty| is
the most important file). You now need to put these files in an
appropriate |texmf| tree. 

When you ask \TeX\ to use a certain class or package, it usually looks
for the necessary files in so-called |texmf| trees. These trees
are simply huge directories that contain these files. By default,
\TeX\ looks for files in three different |texmf| trees:
\begin{itemize}
\item
  The root |texmf| tree, which is usually located at
  |/usr/share/texmf/|, |c:\texmf\|, or\\
  |c:\Program Files\TeXLive\texmf\|.
\item
  The local  |texmf| tree, which is usually located at
  |/usr/local/share/texmf/|, |c:\localtexmf\|, or\\
  |c:\Program Files\TeXLive\texmf-local\|.
\item
  Your personal  |texmf| tree, which is usually located in your home
  directory at |~/texmf/| or |~/Library/texmf/|.   
\end{itemize}

You should install the packages either in the local tree or in
your personal tree, depending on whether you have write access to the
local tree. Installation in the root tree can cause problems, since an
update of the whole \TeX\ installation will replace this whole tree.

Inside whatever |texmf| directory you have chosen, create
the sub-sub-sub-directories
\begin{itemize}
\item
  |texmf/tex/latex/beamer|,
\item
  |texmf/tex/latex/pgf|, and
\item
  |texmf/tex/latex/xcolor|
\end{itemize}
and place all files in these three directories.

Finally, you need to rebuild \TeX's filename database. This done by
running the command  |texhash| or |mktexlsr| (they are
the same). In MiK\TeX, there is a menu option to do this.

\lyxnote
For usage of the \beamer\ class with \LyX, you have to do all of the
above. Then you also have to make \LyX\ aware of the file
|beamer/lyx/layouts/beamer.layout|. To do so, link (or, not
so good in case of later updates, copy) this file to the directory
|.lyx/layouts/| in your home directory. Then use \LyX's Reconfigure
command to make \LyX\ aware of this file.

\vskip1em
For a more detailed explanation of the standard installation process
of packages, you might wish to consult
\href{http://www.ctan.org/installationadvice/}{|http://www.ctan.org/installationadvice/|}.
However, note that the \beamer\ package does not come with a
|.ins| file (simply skip that part).




\subsection{Updating the Installation}

To update your installation from a previous version, simply replace
everything in the directories like |texmf/tex/latex/beamer| with the
files of the new version. The easiest way to do this is to first
delete the old version and then proceed as described above. Sometimes,
there are changes in the syntax of certain command from version to
version. If things no longer work that used to work, you wish to have
a look at the release notes and at the change log.


\subsection{Testing the Installation}

To test your installation, copy the file |beamerexample1.tex|
from the examples subdirectory to some place where you usually
create presentations. Then run the command |pdflatex| several times on
the file and check whether the resulting |beamerexample1.pdf|
looks correct. If so, you are all set.

\lyxnote
To test the \LyX\ installation, create a new file from the
template |generic-ornate-15min-45min.en.lyx|, which is located in the directory
|beamer/solutions/generic-talks|.


% $Header: /cvsroot/latex-beamer/latex-beamer/doc/beamerug-compatibility.tex,v 1.11 2005/05/09 12:41:15 tantau Exp $

% Copyright 2003, 2004 by Till Tantau <tantau@users.sourceforge.net>.
%
% This program can be redistributed and/or modified under the terms
% of the GNU Public License, version 2.

\subsection{Compatibility with Other Packages and Classes}

When using certain packages or classes together with the |beamer|
class, extra options or precautions may be necessary.

\begin{package}{{AlDraTex}}
  Graphics created using AlDraTex must be treated like verbatim
  text. The reason is that DraTex fiddles with catcodes and spaces
  much like verbatim does. So, in order to insert a picture, either
  add the |fragile| option to the frame or use the
  |\defverbatim| command to create a box containing the picture.
\end{package}

\begin{package}{{alltt}}
  Text in an |alltt| environment must be treated like verbatim
  text. So add the |fragile| option to frames containing this
  environment or use |\defverbatim|.
\end{package}

\begin{package}{{amsthm}}
  This package is automatically loaded since \beamer\ uses it for
  typesetting theorems. If you do not wish it to be loaded, which can
  be necessary especially in |article| mode if the package is
  incompatible with the document class, you can use the class option
  |noamsthm| to suppress its loading. See
  Section~\ref{section-theorems} for more details.
\end{package}

\begin{package}{{babel}|[|\declare{|french|}|]|}
  When using the |french| style, certain features that clash with the
  functionality of the \beamer\ class will be turned off. For example,
  enumerations are still produced the way the theme dictates, not the
  way the |french| style does.
\end{package}

\begin{package}{{babel}|[|\declare{|spanish|}|]|}
  \beamernote
  When using the |spanish| style, certain features that clash with the
  functionality of the \beamer\ class will be turned off. In particular,
  the special behaviour of the pointed brackets |<| and |>| is
  deactivated. 

  \articlenote
  To make the characters |<| and |>| active in |article| mode, pass
  the option |activeospeccharacters| to the package
  |beamerbasearticle|. This will lead to
  problems with overlay specifications.
\end{package}

\begin{package}{{color}}
  \beamernote
  The |color| package is automatically loaded by |beamer.cls|. This
  makes it impossible to pass options to |color| in the preamble of
  your document in the normal manner. To pass a \meta{list of options}
  to |color|, you can use the following class option:

  \begin{classoption}{color={\normalfont\meta{list of options}}}
    Causes the \meta{list of options} to be passed on to the |color|
    package. If the \meta{list of options} contains more than one
    option you must enclose it in curly brackets.
  \end{classoption}

  \articlenote
  The |color| package is not loaded automatically if
  |beamerarticle| is loaded with the |noxcolor| option.
\end{package}

\begin{package}{{colortbl}}
  \beamernote
  With newer versions of |xcolor.sty|, you need to pass the option
  |table| to |xcolor.sty| if you wish to use |colortbl|. See the notes
  on |xcolor| below, on how to do this.
\end{package}

\begin{package}{{CJK}}
  \beamernote
  When using the |CJK| package for using Asian fonts, you must use the
  class option \declare{|CJK|}. See |beamerexample4.tex| for an
  example. 
\end{package}

\begin{package}{{deluxetable}}
  \beamernote
  The caption generation facilities of |deluxetable| are
  deactivated. Instead, the caption template is used.
\end{package}

\begin{package}{{DraTex}}
  See |AlDraTex|.
\end{package}

\begin{package}{{enumerate}}
  \articlenote
  This package is loaded automatically in the |presentation| modes, but not
  in the |article| mode. If you use its features, you have to load the
  package ``by hand'' in the |article| mode.
\end{package}

\begin{class}{{foils}}
  If you wish to emulate the |foils| class using \beamer, please see
  Section~\ref{section-foiltex}.
\end{class}

\begin{package}{{fontenc}|[|\declare{|T1|}|]|}
  Use this option only with fonts that have outline fonts available in
  the T1 encoding like Times or the |lmodern| fonts. In a standard
  installation the standard Computer Modern fonts (the fonts Donald
  Knuth originally designed and which are used by default) are
  \emph{not} available in the T1 encoding. Using this  option with
  them will result in very poor rendering of your presentation when
  viewed with \pdf\ viewer applications like Acrobat or |xpdf|. To use
  the Computer Modern fonts with the T1 encoding, use the package
  |lmodern|.  See also Section~\ref{section-font-encoding}.
\end{package}

\begin{package}{{fourier}}
  The package switches to a T1~encoding, but it does not redefine all
  fonts such that outline fonts (non-bitmapped fonts) are used by
  default. For example, the sans-serif text and the typewriter text
  are not replaced. To use outline fonts for these, write
  |\usepackage{lmodern}| \emph{before} including the |fourier|
  package. 
\end{package}

\begin{package}{{HA-prosper}}
  You cannot use this package with \beamer. However, you might try to
  use the package |beamerprosper| instead, see
  Section~\ref{section-prosper}. 
\end{package}

\begin{package}{{hyperref}}
  \beamernote
  The |hyperref| package is automatically loaded by |beamer.cls| and
  certain options are setup. In order pass additional options to
  |hyperref| or to override options, you can use the following class
  option: 

  \begin{classoption}{hyperref={\normalfont\meta{list of options}}}
    Causes the \meta{list of options} to be passed on to the |hyperref|
    package.

    \example |\documentclass[hyperref={bookmarks=false}]{beamer}|
  \end{classoption}

  Alternatively, you can also use the |\hypersetup| command.

  \articlenote
  In the |article| version, you must include |hyperref| manually if
  you want to use it. It is not included automatically.
\end{package}

\begin{package}{{inputenc}|[|\declare{|utf8|}|]|}
  \beamernote
  When using Unicode, you may wish to use one of the following class
  options: 
  \begin{classoption}{ucs}
    Loads the package |ucs| and passes the correct Unicode options to
    |hyperref|. Also, it preloads the Unicode code pages zero and
    one.
  \end{classoption}
  
  \begin{classoption}{utf8}
    Same as the option |ucs|, but also sets the input encoding to
    |utf8|. You could also use the option |ucs| and say
    |\usepackage[utf8]{inputenc}| in the preamble.
  \end{classoption}

  If you use a Unicode character outside the first two code pages
  (which includes the Latin alphabet and the extended Latin alphabet)
  in a section or subsection heading, you have to use the command
  |\PreloadUnicodePage{|\meta{code  page}|}| to give |ucs| a chance to
  preload these code pages. You will know that a character has not
  been preloaded, if you get a message like ``Please insert into
  preamble.'' The code page of a character is given by the unicode
  number of the character divided by 256.
\end{package}

\begin{package}{{listings}}
  \beamernote
  Note that you must treat |lstlisting| environments exactly the same
  way as you would treat |verbatim| environments. When using
  |\defverbatim| that contains a colored |lstlisting|, use the
  |colored| option of |\defverbatim|.
\end{package}

\begin{package}{{msc}}
  \beamernote
  Since this packages uses |pstricks| internally, everything that
  applies to pstricks also applies to |msc|.
\end{package}

\begin{package}{{musixtex}}
  When using  MusiX\TeX\ to typeset musical scores, your document must
  be compiled with |pdfelatex| or |elatex| instead of |pdflatex| or
  |latex|.  

  Inside  a |music| environment, the |\pause| is redefined to match
  MusiX\TeX's definition (a rest during one quarter of a whole). You can
  use the |\beamerpause| command  to create overlays in this
  environment.  
\end{package}

\begin{package}{{pdfpages}}
  Commands like |\includepdf| only work \emph{outside} frames as they
  produce pages ``by themselves.'' You may also wish to say
\begin{verbatim}
\setbeamercolor{background canvas}{bg=} 
\end{verbatim}
  when you use such a command since the background (even a white
  background) will otherwise be printed over the image you try to include.

  \example
\begin{verbatim}
\begin{document}
\begin{frame}
  \titlepage
\end{frame}

{
  \setbeamercolor{background canvas}{bg=} 
  \includepdf{somepdfimages.pdf}
}

\begin{frame}
  A normal frame.
\end{frame}
\end{document}
\end{verbatim}
\end{package}

\begin{package}{{\normalfont\meta{professional font package}}}
  \beamernote
  If you use a professional font package, \beamer's internal
  redefinition of how variables are typeset may interfere with the
  font package's superior way of typesetting them. In this case, you
  should use the class option |professionalfont| to suppress any font
  substitution. See Section~\ref{section-substition} for details.
\end{package}

\begin{class}{{prosper}}
  If you wish to (partly) emulate the |prosper| class using \beamer,
  please see Section~\ref{section-prosper}.
\end{class}

\begin{package}{{pstricks}}
  You should add the option |xcolor=pst| to make |xcolor| aware of the
  fact that you are using |pstricks|.
\end{package}

\begin{class}{{seminar}}
  If you wish to emulate the |seminar| class using \beamer, please see
  Section~\ref{section-seminar}.
\end{class}

\begin{package}{{texpower}}
  You cannot use this package with \beamer. However, you might try to
  use the package |beamertexpower| instead, see
  Section~\ref{section-texpower}.  
\end{package}

\begin{package}{{textpos}}
  \beamernote
  \beamer\ automatically installs a white background behind
  everything, unless you install a different background
  template. Because of this, you must use the |overlay| option when
  using |textpos|, so that it will place boxes \emph{before}
  everything. Alternatively, you can install an empty background
  template, but this may result in an incorrect display in certain
  situtations with older versions of the Acrobat Reader. 
\end{package}

\begin{package}{{ucs}}
  See |\usepackage[utf8]{inputenc}|.
\end{package}


\begin{package}{{xcolor}}
  \beamernote
  The |xcolor| package is automatically loaded by |beamer.cls|. The
  same applies as to |color|.

  \begin{classoption}{xcolor={\normalfont\meta{list of options}}}
    Causes the \meta{list of options} to be passed on to the |xcolor|
    package.
  \end{classoption}

  When using \beamer\ together with the |pstricks| package, be sure to
  pass the |xcolor=pst| option to \beamer\ (and hence to |xcolor|).

  \articlenote
  The |color| package is not loaded automatically if
  |beamerarticle| is loaded with the |noxcolor| option.
\end{package}




%%% Local Variables: 
%%% mode: latex
%%% TeX-master: "beameruserguide"
%%% End: 


% $Header: /cvsroot/latex-beamer/latex-beamer/doc/beamerug-license.tex,v 1.3 2004/10/07 20:53:05 tantau Exp $

% Copyright 2003, 2004 by Till Tantau <tantau@users.sourceforge.net>.
%
% This program can be redistributed and/or modified under the terms
% of the GNU Public License, version 2.


\subsection{License: The GNU Public License, Version 2}
\label{section-license}

The \beamer\ class is distributed under the \textsc{gnu} public
license, version 2. In detail, this means the following (the following
text is copyrighted by the Free Software Foundation):

\subsubsection{Preamble}

The licenses for most software are designed to take away your freedom to
share and change it.  By contrast, the \textsc{gnu} General Public License is
intended to guarantee your freedom to share and change free software---to
make sure the software is free for all its users.  This General Public
License applies to most of the Free Software Foundation's software and to
any other program whose authors commit to using it.  (Some other Free
Software Foundation software is covered by the \textsc{gnu} Library General Public
License instead.)  You can apply it to your programs, too.

When we speak of free software, we are referring to freedom, not price.
Our General Public Licenses are designed to make sure that you have the
freedom to distribute copies of free software (and charge for this service
if you wish), that you receive source code or can get it if you want it,
that you can change the software or use pieces of it in new free programs;
and that you know you can do these things.

To protect your rights, we need to make restrictions that forbid anyone to
deny you these rights or to ask you to surrender the rights.  These
restrictions translate to certain responsibilities for you if you
distribute copies of the software, or if you modify it.

For example, if you distribute copies of such a program, whether gratis or
for a fee, you must give the recipients all the rights that you have.  You
must make sure that they, too, receive or can get the source code.  And
you must show them these terms so they know their rights.

We protect your rights with two steps: (1) copyright the software, and (2)
offer you this license which gives you legal permission to copy,
distribute and/or modify the software.

Also, for each author's protection and ours, we want to make certain that
everyone understands that there is no warranty for this free software.  If
the software is modified by someone else and passed on, we want its
recipients to know that what they have is not the original, so that any
problems introduced by others will not reflect on the original authors'
reputations.

Finally, any free program is threatened constantly by software patents.
We wish to avoid the danger that redistributors of a free program will
individually obtain patent licenses, in effect making the program
proprietary.  To prevent this, we have made it clear that any patent must
be licensed for everyone's free use or not licensed at all.

The precise terms and conditions for copying, distribution and
modification follow.

\subsubsection{Terms and Conditions For Copying, Distribution and
  Modification}

\begin{enumerate}

\addtocounter{enumi}{-1}

\item 

This License applies to any program or other work which contains a notice
placed by the copyright holder saying it may be distributed under the
terms of this General Public License.  The ``Program'', below, refers to
any such program or work, and a ``work based on the Program'' means either
the Program or any derivative work under copyright law: that is to say, a
work containing the Program or a portion of it, either verbatim or with
modifications and/or translated into another language.  (Hereinafter,
translation is included without limitation in the term ``modification''.)
Each licensee is addressed as ``you''.

Activities other than copying, distribution and modification are not
covered by this License; they are outside its scope.  The act of
running the Program is not restricted, and the output from the Program
is covered only if its contents constitute a work based on the
Program (independent of having been made by running the Program).
Whether that is true depends on what the Program does.

\item You may copy and distribute verbatim copies of the Program's source
  code as you receive it, in any medium, provided that you conspicuously
  and appropriately publish on each copy an appropriate copyright notice
  and disclaimer of warranty; keep intact all the notices that refer to
  this License and to the absence of any warranty; and give any other
  recipients of the Program a copy of this License along with the Program.

You may charge a fee for the physical act of transferring a copy, and you
may at your option offer warranty protection in exchange for a fee.

\item

You may modify your copy or copies of the Program or any portion
of it, thus forming a work based on the Program, and copy and
distribute such modifications or work under the terms of Section 1
above, provided that you also meet all of these conditions:

\begin{enumerate}

\item 

You must cause the modified files to carry prominent notices stating that
you changed the files and the date of any change.

\item

You must cause any work that you distribute or publish, that in
whole or in part contains or is derived from the Program or any
part thereof, to be licensed as a whole at no charge to all third
parties under the terms of this License.

\item
If the modified program normally reads commands interactively
when run, you must cause it, when started running for such
interactive use in the most ordinary way, to print or display an
announcement including an appropriate copyright notice and a
notice that there is no warranty (or else, saying that you provide
a warranty) and that users may redistribute the program under
these conditions, and telling the user how to view a copy of this
License.  (Exception: if the Program itself is interactive but
does not normally print such an announcement, your work based on
the Program is not required to print an announcement.)

\end{enumerate}


These requirements apply to the modified work as a whole.  If
identifiable sections of that work are not derived from the Program,
and can be reasonably considered independent and separate works in
themselves, then this License, and its terms, do not apply to those
sections when you distribute them as separate works.  But when you
distribute the same sections as part of a whole which is a work based
on the Program, the distribution of the whole must be on the terms of
this License, whose permissions for other licensees extend to the
entire whole, and thus to each and every part regardless of who wrote it.

Thus, it is not the intent of this section to claim rights or contest
your rights to work written entirely by you; rather, the intent is to
exercise the right to control the distribution of derivative or
collective works based on the Program.

In addition, mere aggregation of another work not based on the Program
with the Program (or with a work based on the Program) on a volume of
a storage or distribution medium does not bring the other work under
the scope of this License.

\item
You may copy and distribute the Program (or a work based on it,
under Section 2) in object code or executable form under the terms of
Sections 1 and 2 above provided that you also do one of the following:

\begin{enumerate}

\item

Accompany it with the complete corresponding machine-readable
source code, which must be distributed under the terms of Sections
1 and 2 above on a medium customarily used for software interchange; or,

\item

Accompany it with a written offer, valid for at least three
years, to give any third party, for a charge no more than your
cost of physically performing source distribution, a complete
machine-readable copy of the corresponding source code, to be
distributed under the terms of Sections 1 and 2 above on a medium
customarily used for software interchange; or,

\item

Accompany it with the information you received as to the offer
to distribute corresponding source code.  (This alternative is
allowed only for noncommercial distribution and only if you
received the program in object code or executable form with such
an offer, in accord with Subsubsection b above.)

\end{enumerate}


The source code for a work means the preferred form of the work for
making modifications to it.  For an executable work, complete source
code means all the source code for all modules it contains, plus any
associated interface definition files, plus the scripts used to
control compilation and installation of the executable.  However, as a
special exception, the source code distributed need not include
anything that is normally distributed (in either source or binary
form) with the major components (compiler, kernel, and so on) of the
operating system on which the executable runs, unless that component
itself accompanies the executable.

If distribution of executable or object code is made by offering
access to copy from a designated place, then offering equivalent
access to copy the source code from the same place counts as
distribution of the source code, even though third parties are not
compelled to copy the source along with the object code.

\item
You may not copy, modify, sublicense, or distribute the Program
except as expressly provided under this License.  Any attempt
otherwise to copy, modify, sublicense or distribute the Program is
void, and will automatically terminate your rights under this License.
However, parties who have received copies, or rights, from you under
this License will not have their licenses terminated so long as such
parties remain in full compliance.

\item
You are not required to accept this License, since you have not
signed it.  However, nothing else grants you permission to modify or
distribute the Program or its derivative works.  These actions are
prohibited by law if you do not accept this License.  Therefore, by
modifying or distributing the Program (or any work based on the
Program), you indicate your acceptance of this License to do so, and
all its terms and conditions for copying, distributing or modifying
the Program or works based on it.

\item
Each time you redistribute the Program (or any work based on the
Program), the recipient automatically receives a license from the
original licensor to copy, distribute or modify the Program subject to
these terms and conditions.  You may not impose any further
restrictions on the recipients' exercise of the rights granted herein.
You are not responsible for enforcing compliance by third parties to
this License.

\item
If, as a consequence of a court judgment or allegation of patent
infringement or for any other reason (not limited to patent issues),
conditions are imposed on you (whether by court order, agreement or
otherwise) that contradict the conditions of this License, they do not
excuse you from the conditions of this License.  If you cannot
distribute so as to satisfy simultaneously your obligations under this
License and any other pertinent obligations, then as a consequence you
may not distribute the Program at all.  For example, if a patent
license would not permit royalty-free redistribution of the Program by
all those who receive copies directly or indirectly through you, then
the only way you could satisfy both it and this License would be to
refrain entirely from distribution of the Program.

If any portion of this section is held invalid or unenforceable under
any particular circumstance, the balance of the section is intended to
apply and the section as a whole is intended to apply in other
circumstances.

It is not the purpose of this section to induce you to infringe any
patents or other property right claims or to contest validity of any
such claims; this section has the sole purpose of protecting the
integrity of the free software distribution system, which is
implemented by public license practices.  Many people have made
generous contributions to the wide range of software distributed
through that system in reliance on consistent application of that
system; it is up to the author/donor to decide if he or she is willing
to distribute software through any other system and a licensee cannot
impose that choice.

This section is intended to make thoroughly clear what is believed to
be a consequence of the rest of this License.

\item
If the distribution and/or use of the Program is restricted in
certain countries either by patents or by copyrighted interfaces, the
original copyright holder who places the Program under this License
may add an explicit geographical distribution limitation excluding
those countries, so that distribution is permitted only in or among
countries not thus excluded.  In such case, this License incorporates
the limitation as if written in the body of this License.

\item
The Free Software Foundation may publish revised and/or new versions
of the General Public License from time to time.  Such new versions will
be similar in spirit to the present version, but may differ in detail to
address new problems or concerns.

Each version is given a distinguishing version number.  If the Program
specifies a version number of this License which applies to it and ``any
later version'', you have the option of following the terms and conditions
either of that version or of any later version published by the Free
Software Foundation.  If the Program does not specify a version number of
this License, you may choose any version ever published by the Free Software
Foundation.

\item
If you wish to incorporate parts of the Program into other free
programs whose distribution conditions are different, write to the author
to ask for permission.  For software which is copyrighted by the Free
Software Foundation, write to the Free Software Foundation; we sometimes
make exceptions for this.  Our decision will be guided by the two goals
of preserving the free status of all derivatives of our free software and
of promoting the sharing and reuse of software generally.

\end{enumerate}

\subsubsection{No Warranty}

\begin{enumerate}

\addtocounter{enumi}{9}

\item
Because the program is licensed free of charge, there is no warranty
for the program, to the extent permitted by applicable law.  Except when
otherwise stated in writing the copyright holders and/or other parties
provide the program ``as is'' without warranty of any kind, either expressed
or implied, including, but not limited to, the implied warranties of
merchantability and fitness for a particular purpose.  The entire risk as
to the quality and performance of the program is with you.  Should the
program prove defective, you assume the cost of all necessary servicing,
repair or correction.

\item
In no event unless required by applicable law or agreed to in writing
will any copyright holder, or any other party who may modify and/or
redistribute the program as permitted above, be liable to you for damages,
including any general, special, incidental or consequential damages arising
out of the use or inability to use the program (including but not limited
to loss of data or data being rendered inaccurate or losses sustained by
you or third parties or a failure of the program to operate with any other
programs), even if such holder or other party has been advised of the
possibility of such damages.
\end{enumerate}

%%% Local Variables: 
%%% mode: latex
%%% TeX-master: "beameruserguide"
%%% End: 


%%% Local Variables: 
%%% mode: latex
%%% TeX-master: "beameruserguide"
%%% End: 
