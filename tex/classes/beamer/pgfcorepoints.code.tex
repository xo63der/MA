\ProvidesFileRCS $Header: /cvsroot/latex-beamer/pgf/generic/pgf/basiclayer/pgfcorepoints.code.tex,v 1.1 2005/06/11 17:13:09 tantau Exp $

% Copyright 2005 by Till Tantau <tantau@cs.tu-berlin.de>.
%
% This program can be redistributed and/or modified under the terms
% of the GNU Public License, version 2.

\newdimen\pgf@picminx
\newdimen\pgf@picmaxx
\newdimen\pgf@picminy
\newdimen\pgf@picmaxy

\newdimen\pgf@pathminx
\newdimen\pgf@pathmaxx
\newdimen\pgf@pathminy
\newdimen\pgf@pathmaxy

\newif\ifpgf@relevantforpicturesize

\def\pgf@process#1{{#1\global\pgf@x=\pgf@x\global\pgf@y=\pgf@y}}

\newdimen\pgf@tempdim
\def\pgf@setlength#1#2{% these will be used only when \nullfont is active
  \begingroup% keep font setting local
    \pgf@selectfontorig% restore font
    \pgf@setlengthorig\pgf@tempdim{#2}% calculate dimension (possibly using calc)
    \global\pgf@tempdim\pgf@tempdim% make dimension global
  \endgroup%
  #1=\pgf@tempdim\relax}
\def\pgf@addtolength#1#2{%
  \begingroup% keep font setting local
    \pgf@selectfontorig% restore font
    \pgf@tempdim#1\relax%
    \pgf@addtolengthorig\pgf@tempdim{#2}% calculate dimension (possibly using calc)
    \global\pgf@tempdim\pgf@tempdim% make dimension global
  \endgroup%
  #1=\pgf@tempdim\relax}
\newcounter{pgf@tempcount}
\def\pgf@setcounter#1#2{%
  \setcounter{pgf@tempcount}{#2}% makes change global!
  \csname c@#1\endcsname=\c@pgf@tempcount\relax}
\def\pgf@selectfont{\pgf@selectfontorig\nullfont}



% Return a point
%  
% #1 = x-coordinate of the point 
% #2 = y-coordinate of the point 
%  
% x = #1
% y = #2
%
% Example:
%
% \pgfpathmoveto{\pgfpoint{2pt}{3cm}}

\def\pgfpoint#1#2{%
  \setlength\pgf@x{#1}%
  \setlength\pgf@y{#2}\ignorespaces}

% Return the origin.
%  
% x = 0
% y = 0
%
% Example:
%
% \pgfpathmoveto{\pgfpointorigin}

\def\pgfpointorigin{\pgfpoint{0pt}{0pt}\ignorespaces}



% Return a transformed point
%  
% #1 = a point
%
% Description:
%
% This command applies pgf's current transformation matrix to the
% given point. Normally, this is done automatically by commands like
% lineto or moveto, but sometimes you may wish to access a transformed
% point yourself. For example, this command is useful for a low level
% coordinate system shift:
%
% Example:
%
% \begin{pgflowleveltransformshiftscope}{\pgfpointtransformed{\pgfpointorigin}}
%   \pgfbox[center,center]{Hi!}
% \end{pgflowleveltransformshiftscope}

\def\pgfpointtransformed#1{%
  \pgf@process{%
    #1%
    \pgf@pos@transform{\pgf@x}{\pgf@y}%
  }%
}


% Return the difference vector of two points.
%  
% #1 = start of vector  
% #2 = end of vector  
%  
% x = x-component of difference 
% y = y-component of difference 
%
% Example:
%
% \pgfpathmoveto{\pgfpointdiff{\pgfpointxy{1}{1}}{\pgfpointxy{2}{3}}}

\def\pgfpointdiff#1#2{%
  \pgf@process{#1}%
  \pgf@xa=\pgf@x%
  \pgf@ya=\pgf@y%
  \pgf@process{#2}%
  \advance\pgf@x by-\pgf@xa\relax%
  \advance\pgf@y by-\pgf@ya\relax\ignorespaces}

% Add two vectors. 
%  
% #1 = first vector  
% #2 = second vector  
%  
% x = x-component of addition
% y = y-component of addition
%
% Example:
%
% \pgfpathmoveto{\pgfpointadd{\pgfpointxy{0}{1}}{\pgfpointxy{2}{3}}}

\def\pgfpointadd#1#2{%
  \pgf@process{#1}%
  \pgf@xa=\pgf@x%
  \pgf@ya=\pgf@y%
  \pgf@process{#2}%
  \advance\pgf@x by\pgf@xa%
  \advance\pgf@y by\pgf@ya}



% Multiply a vector by a factor.
%  
% #1 = factor  
% #2 = vector  
%
% Example:
%
% \pgfpointscale{2}{\pgfpointxy{0}{1}}

\def\pgfpointscale#1#2{%
  \pgf@process{#2}%
  \pgf@x=#1\pgf@x%
  \pgf@y=#1\pgf@y%
}


% Returns point on a line from #2 to #3 at time #1.
%  
% #1 = a time, where 0 is the start and 1 is the end
% #2 = start point
% #3 = end point 
%  
% x = x-component of #1*start + (1-#1)*end
% y = y-component of #1*start + (1-#1)*end 
%
% Example:
%
% % Middle of (1,1) and (2,3)
% \pgfpathmoveto{\pgfpointlineattime{0.5}{\pgfpointxy{0}{1}}{\pgfpointxy{2}{3}}}

\def\pgfpointlineattime#1#2#3{%
  \pgf@process{#3}%
  \pgf@xa=\pgf@x%
  \pgf@ya=\pgf@y%
  \pgf@process{#2}%
  \advance\pgf@xa by-\pgf@x\relax%
  \advance\pgf@ya by-\pgf@y\relax%
  \advance\pgf@x by #1\pgf@xa\relax%
  \advance\pgf@y by #1\pgf@ya\relax%  
  \ignorespaces}


% Move point #2 #1 many units in the direction of #3.
%  
% #1 = a distance
% #2 = start point
% #3 = end point 
%  
% x = x-component of start + #1*(normalise(end-start))
% y = y-component of start + #1*(normalise(end-start))
%
% Example:
%
%
% \pgfpathmoveto{\pgfpointlineatdistance{2pt}{\pgfpointxy{0}{1}}{\pgfpointxy{2}{3}}}
% \pgfpathlineto{\pgfpointlineatdistance{3pt}{\pgfpointxy{2}{3}}{\pgfpointxy{0}{1}}}

\def\pgfpointlineatdistance#1#2#3{%
  \pgf@process{#2}%
  \pgf@xb=\pgf@x\relax% xb/yb = start point
  \pgf@yb=\pgf@y\relax%
  \pgf@process{#3}%
  \advance\pgf@x by-\pgf@xb\relax%
  \advance\pgf@y by-\pgf@yb\relax%
  \pgf@process{\pgfpointnormalised{}}% x/y = normalised vector
  \setlength\pgf@xa{#1}%
  \pgf@ya=\pgf@xa\relax%
  \pgf@xa=\pgf@sys@tonumber{\pgf@x}\pgf@xa%
  \pgf@ya=\pgf@sys@tonumber{\pgf@y}\pgf@ya%
  \pgf@x=\pgf@xb\relax%
  \pgf@y=\pgf@yb\relax%
  \advance\pgf@x by\pgf@xa\relax%
  \advance\pgf@y by\pgf@ya\relax%
  \ignorespaces}


% Returns point on a curve from #2 to #5 with controls #3 and #4 at time #1.
%  
% #1 = a time
% #2 = start point
% #3 = first control point
% #4 = second control point
% #5 = end point 
%  
% x = x-component of place on the curve at time t
% y = y-component of place on the curve at time t
%
% Additionally, (\pgf@xa,\pgf@ya) and (\pgf@xb,\pgf@yb) will be on a
% tangent to the point on the curve (this can be useful for computing
% a label rotation).
%
% Example:
%
% % Middle of (1,1) and (2,3)
% \pgfpathmoveto{\pgfpointcurveattime{0.5}{\pgfpointxy{0}{1}}{\pgfpointxy{1}{1}}{\pgfpointxy{1}{1}}{\pgfpointxy{2}{3}}}

\def\pgfpointcurveattime#1#2#3#4#5{%
  \def\pgf@time@s{#1}%
  \pgf@x=#1pt%
  \pgf@x=-\pgf@x%
  \advance\pgf@x by 1pt%
  \edef\pgf@time@t{\pgf@sys@tonumber{\pgf@x}}%
  \pgf@process{#5}%
  \pgf@xc=\pgf@x%
  \pgf@yc=\pgf@y%
  \pgf@process{#4}%
  \pgf@xb=\pgf@x%
  \pgf@yb=\pgf@y%
  \pgf@process{#3}%
  \pgf@xa=\pgf@x%
  \pgf@ya=\pgf@y%
  \pgf@process{#2}%
  % First iteration:
  \pgf@x=\pgf@time@t\pgf@x\advance\pgf@x by\pgf@time@s\pgf@xa%
  \pgf@y=\pgf@time@t\pgf@y\advance\pgf@y by\pgf@time@s\pgf@ya%
  \pgf@xa=\pgf@time@t\pgf@xa\advance\pgf@xa by\pgf@time@s\pgf@xb%
  \pgf@ya=\pgf@time@t\pgf@ya\advance\pgf@ya by\pgf@time@s\pgf@yb%
  \pgf@xb=\pgf@time@t\pgf@xb\advance\pgf@xb by\pgf@time@s\pgf@xc%
  \pgf@yb=\pgf@time@t\pgf@yb\advance\pgf@yb by\pgf@time@s\pgf@yc%
  % Second iteration:
  \pgf@x=\pgf@time@t\pgf@x\advance\pgf@x by\pgf@time@s\pgf@xa%
  \pgf@y=\pgf@time@t\pgf@y\advance\pgf@y by\pgf@time@s\pgf@ya%
  \pgf@xa=\pgf@time@t\pgf@xa\advance\pgf@xa by\pgf@time@s\pgf@xb%
  \pgf@ya=\pgf@time@t\pgf@ya\advance\pgf@ya by\pgf@time@s\pgf@yb%
  % Save x/y
  \pgf@xb=\pgf@x%
  \pgf@yb=\pgf@y%
  % Third iteration:
  \pgf@x=\pgf@time@t\pgf@x\advance\pgf@x by\pgf@time@s\pgf@xa%
  \pgf@y=\pgf@time@t\pgf@y\advance\pgf@y by\pgf@time@s\pgf@ya%
}






% Internal registers
\newdimen\pgf@xx
\newdimen\pgf@xy
\newdimen\pgf@yx
\newdimen\pgf@yy
\newdimen\pgf@zx
\newdimen\pgf@zy



% Store a value in polar-coordinates
%
% #1 = a degree
% #2 = a radius
%
% x = #2 * cos(#1)
% y = #2 * sin(#2)
% 
% Example:
%
% % Create a slanted rectangle
%
% \pgfpathmoveto{\pgfpointpolar{30}{1cm}}

\def\pgfpointpolar#1#2{%
  \pgfsincos{#1}%
  \in@/{#2}%
  \ifin@%
    \pgf@polar@#2\@@%
  \else%
    \pgf@polar@#2/#2\@@%
  \fi%  
  \pgf@xa=\pgf@sys@tonumber{\pgf@x}\pgf@xa%
  \pgf@ya=\pgf@sys@tonumber{\pgf@y}\pgf@ya%
  \pgf@y=\pgf@xa\relax%
  \pgf@x=\pgf@ya\ignorespaces}

\def\pgf@polar@#1/#2\@@{%
  \setlength{\pgf@xa}{#2}%
  \setlength{\pgf@ya}{#1}%  
}



% Store the vector #1 * x-vec + #2 * y-vec
%
% #1 = a factor for the x-vector
% #2 = a factor fot the y-vector
%
% x = x-component of result vector
% y = y-component of result vector
%
% Description:
% 
% This command can be used to create a new coordinate system
% without using the rotate/translate/scale commands. This
% may be useful, if you do not want arrows and line width to
% be scaled/transformed together with the coordinate system.
% 
% Example:
%
% % Create a slanted rectangle
%
% \pgfsetxvec{\pgfpoint{1cm}{1cm}}
% \pgfsetyvec{\pgfpoint{0cm}{1cm}}
% 
% \pgfpathmoveto{\pgfpointxy{0}{0}}
% \pgfpathlineto{\pgfpointxy{1}{0}}
% \pgfpathlineto{\pgfpointxy{1}{1}}
% \pgfpathlineto{\pgfpointxy{0}{1}}
% \pgfclosestroke

\def\pgfpointxy#1#2{\pgf@x=#1\pgf@xx%
  \advance\pgf@x by #2\pgf@yx%
  \pgf@y=#1\pgf@xy%
  \advance\pgf@y by #2\pgf@yy}


% Store the vector #1 * x-vec + #2 * y-vec + #3 * z-vec
%
% #1 = a factor for the x-vector
% #2 = a factor fot the y-vector
% #3 = a factor fot the z-vector
%
% x = x-component of result vector
% y = y-component of result vector
%
%
% Description:
% 
% This command allows you to use a 3d coordinate system.
% 
%
% Example:
%
% % Draw a cubus
% 
% \pgfline{\pgfpointxyz{0}{0}{0}}{\pgfpointxyz{0}{0}{1}}
% \pgfline{\pgfpointxyz{0}{1}{0}}{\pgfpointxyz{0}{1}{1}}
% \pgfline{\pgfpointxyz{1}{0}{0}}{\pgfpointxyz{1}{0}{1}}
% \pgfline{\pgfpointxyz{1}{1}{0}}{\pgfpointxyz{1}{1}{1}}
% \pgfline{\pgfpointxyz{0}{0}{0}}{\pgfpointxyz{0}{1}{0}}
% \pgfline{\pgfpointxyz{0}{0}{1}}{\pgfpointxyz{0}{1}{1}}
% \pgfline{\pgfpointxyz{1}{0}{0}}{\pgfpointxyz{1}{1}{0}}
% \pgfline{\pgfpointxyz{1}{0}{1}}{\pgfpointxyz{1}{1}{1}}
% \pgfline{\pgfpointxyz{0}{0}{0}}{\pgfpointxyz{1}{0}{0}}
% \pgfline{\pgfpointxyz{0}{0}{1}}{\pgfpointxyz{1}{0}{1}}
% \pgfline{\pgfpointxyz{0}{1}{0}}{\pgfpointxyz{1}{1}{0}}
% \pgfline{\pgfpointxyz{0}{1}{1}}{\pgfpointxyz{1}{1}{1}}

\def\pgfpointxyz#1#2#3{%
  \pgf@x=#1\pgf@xx%
  \advance\pgf@x by #2\pgf@yx%
  \advance\pgf@x by #3\pgf@zx%
  \pgf@y=#1\pgf@xy%
  \advance\pgf@y by #2\pgf@yy%
  \advance\pgf@y by #3\pgf@zy}




% Set the x-vector
%
% #1 = a point the is the new x-vector
%
% Example:
%
% \pgfsetxvec{\pgfpoint{1cm}{0cm}}

\def\pgfsetxvec#1{%
  \pgf@process{#1}%
  \pgf@xx=\pgf@x%
  \pgf@xy=\pgf@y%
  \ignorespaces}


% Set the y-vector
%
% #1 = a point the is the new y-vector
%
% Example:
%
% \pgfsetyvec{\pgfpoint{0cm}{1cm}}

\def\pgfsetyvec#1{%
  \pgf@process{#1}%
  \pgf@yx=\pgf@x%
  \pgf@yy=\pgf@y%
  \ignorespaces}


% Set the z-vector
%
% #1 = a point the is the new z-vector
%
% Example:
%
% \pgfsetzvec{\pgfpoint{-0.385cm}{-0.385cm}}

\def\pgfsetzvec#1{%
  \pgf@process{#1}%
  \pgf@zx=\pgf@x%
  \pgf@zy=\pgf@y%
  \ignorespaces}



% Default values
\pgfsetxvec{\pgfpoint{1cm}{0cm}}
\pgfsetyvec{\pgfpoint{0cm}{1cm}}
\pgfsetzvec{\pgfpoint{-0.385cm}{-0.385cm}}




% Normalise a point.
%
% #1 = point with coordinates (a,b)
%
% x  = a/\sqrt(a*a+b*b)
% y  = b/\sqrt(a*a+b*b)
%
% Example:
%
% \pgfpointnormalised{\pgfpointxy{2}{1}}

\def\pgfpointnormalised#1{%
  \pgf@process{#1}%
  \pgf@xa=\pgf@x%
  \pgf@ya=\pgf@y%
  \ifdim\pgf@x<0pt\relax% move into first quadrant
    \pgf@x=-\pgf@x%
  \fi%
  \ifdim\pgf@y<0pt\relax%
    \pgf@y=-\pgf@y%
  \fi%
  \ifdim\pgf@x>\pgf@y% x > y
    % make point small
    \c@pgf@counta=\pgf@x%
    \divide\c@pgf@counta by 65536\relax%
    \ifnum\c@pgf@counta=0\relax%
      \c@pgf@counta=1\relax%
    \fi%
    \divide\pgf@x by\c@pgf@counta%
    \divide\pgf@y by\c@pgf@counta%
    \divide\pgf@xa by\c@pgf@counta%
    \divide\pgf@ya by\c@pgf@counta%
    % ok.
    \pgf@x=.125\pgf@x%
    \pgf@y=.125\pgf@y%
    \c@pgf@counta=\pgf@x%
    \c@pgf@countb=\pgf@y%
    \multiply\c@pgf@countb by 100%
    \ifnum\c@pgf@counta<64\relax%
      \pgf@x=1pt\relax%
      \pgf@y=0pt\relax%
    \else%
      \divide\c@pgf@countb by \c@pgf@counta%
      \pgf@setmath{x}{\csname pgf@cosfrac\the\c@pgf@countb\endcsname}%
      \pgf@xc=8192pt%
      \divide\pgf@xc by\c@pgf@counta%
      \pgf@y=\pgf@sys@tonumber{\pgf@xc}\pgf@ya%
      \pgf@y=\pgf@sys@tonumber{\pgf@x}\pgf@y%
    \fi%
    \ifdim\pgf@xa<0pt%
      \pgf@x=-\pgf@x%
    \fi%
  \else% x <= y
    % make point small
    \c@pgf@counta=\pgf@y%
    \divide\c@pgf@counta by 65536\relax%
    \ifnum\c@pgf@counta=0\relax%
      \c@pgf@counta=1\relax%
    \fi%
    \divide\pgf@x by\c@pgf@counta%
    \divide\pgf@y by\c@pgf@counta%
    \divide\pgf@xa by\c@pgf@counta%
    \divide\pgf@ya by\c@pgf@counta%
    % ok.
    \pgf@x=.125\pgf@x%
    \pgf@y=.125\pgf@y%
    \c@pgf@counta=\pgf@y%
    \c@pgf@countb=\pgf@x%
    \multiply\c@pgf@countb by 100%
    \ifnum\c@pgf@counta<64\relax%
      \pgf@y=1pt\relax%
      \pgf@x=0pt\relax%
    \else%
      \divide\c@pgf@countb by \c@pgf@counta%
      \pgf@setmath{y}{\csname pgf@cosfrac\the\c@pgf@countb\endcsname}%
      \pgf@xc=8192pt%
      \divide\pgf@xc by\c@pgf@counta%
      \pgf@x=\pgf@sys@tonumber{\pgf@xc}\pgf@xa%
      \pgf@x=\pgf@sys@tonumber{\pgf@y}\pgf@x%
    \fi%
    \ifdim\pgf@ya<0pt%
      \pgf@y=-\pgf@y%
    \fi%
  \fi\ignorespaces}





% A point on a rectangle in a certain direction.
%
% #1 = a point pointing in some direction (length should be about 1pt,
%      but need not be exact)
% #2 = upper right corner of a rectangle centered at the origin
%
% Returns the intersection of a line starting at the origin going in
% the given direction and the rectangle's border.
%
% Example:
%
% \pgfpointborderrectangle{\pgfpointnormalised{\pgfpointxy{2}{1}}
%   {\pgfpoint{1cm}{2cm}}

\def\pgfpointborderrectangle#1#2{%
  \pgf@process{#2}%
  \pgf@xb=\pgf@x%
  \pgf@yb=\pgf@y%
  \pgf@process{#1}%
  % Ok, let's find out about the direction:
  \pgf@xa=\pgf@x%
  \pgf@ya=\pgf@y%
  \ifnum\pgf@xa<0\relax% move into first quadrant
    \pgf@x=-\pgf@x%
  \fi%
  \ifnum\pgf@ya<0\relax%
    \pgf@y=-\pgf@y%
  \fi%
  \pgf@xc=.125\pgf@x%
  \pgf@yc=.125\pgf@y%
  \c@pgf@counta=\pgf@xc%
  \c@pgf@countb=\pgf@yc%
  \ifnum\c@pgf@countb<\c@pgf@counta%
    \ifnum\c@pgf@counta<255\relax%
      \pgf@y=\pgf@yb\relax%
      \pgf@x=0pt\relax%
    \else%
      \pgf@xc=8192pt%
      \divide\pgf@xc by\c@pgf@counta% \pgf@xc = 1/\pgf@x
      \pgf@y=\pgf@sys@tonumber{\pgf@xc}\pgf@y%
      \pgf@y=\pgf@sys@tonumber{\pgf@xb}\pgf@y%
      \ifnum\pgf@y<\pgf@yb%
        \pgf@x=\pgf@xb%
      \else% rats, calculate intersection on upper side
        \ifnum\c@pgf@countb<255\relax%
          \pgf@x=\pgf@xb\relax%
          \pgf@y=0pt\relax%
        \else%
          \pgf@yc=8192pt%
          \divide\pgf@yc by\c@pgf@countb% \pgf@xc = 1/\pgf@x
          \pgf@x=\pgf@sys@tonumber{\pgf@yc}\pgf@x%
          \pgf@x=\pgf@sys@tonumber{\pgf@yb}\pgf@x%
          \pgf@y=\pgf@yb%
        \fi%
      \fi%  
    \fi%
  \else%
    \ifnum\c@pgf@countb<255\relax%
      \pgf@x=\pgf@xb\relax%
      \pgf@y=0pt\relax%
    \else%
      \pgf@yc=8192pt%
      \divide\pgf@yc by\c@pgf@countb% \pgf@xc = 1/\pgf@x
      \pgf@x=\pgf@sys@tonumber{\pgf@yc}\pgf@x%
      \pgf@x=\pgf@sys@tonumber{\pgf@yb}\pgf@x%
      \ifnum\pgf@x<\pgf@xb%
        \pgf@y=\pgf@yb%
      \else%
        \ifnum\c@pgf@counta<255\relax%
          \pgf@y=\pgf@yb\relax%
          \pgf@x=0pt\relax%
        \else%
          \pgf@xc=8192pt%
          \divide\pgf@xc by\c@pgf@counta% \pgf@xc = 1/\pgf@x
          \pgf@y=\pgf@sys@tonumber{\pgf@xc}\pgf@y%
          \pgf@y=\pgf@sys@tonumber{\pgf@xb}\pgf@y%
          \pgf@x=\pgf@xb%
        \fi%
      \fi%  
    \fi%
  \fi%  
  \ifnum\pgf@xa<0\relax\pgf@x=-\pgf@x\fi%
  \ifnum\pgf@ya<0\relax\pgf@y=-\pgf@y\fi%    
}




% An approximation to a point on an ellipse in a certain
% direction. Will be exact only if the ellipse is a circle. 
%
% #1 = a point pointing in some direction
% #2 = upper right corner of a bounding box for the ellipse
%
% Returns the intersection of a line starting at the origin going in
% the given direction and the ellipses border.
%
% Example:
%
% \pgfpointborderellipse{\pgfpointnormalised{\pgfpointxy{2}{1}}
%   {\pgfpoint{1cm}{2cm}}

\def\pgfpointborderellipse#1#2{%
  \pgf@process{#2}%
  \pgf@xa=\pgf@x%
  \pgf@ya=\pgf@y%
  \ifdim\pgf@xa=\pgf@ya% circle. that's easy!
    \pgf@process{\pgfpointnormalised{#1}}%
    \pgf@x=\pgf@sys@tonumber{\pgf@xa}\pgf@x%
    \pgf@y=\pgf@sys@tonumber{\pgf@xa}\pgf@y%
  \else%
    \ifdim\pgf@xa<\pgf@ya%
      % Ok, first, let's compute x/y:
      \c@pgf@countb=\pgf@ya%
      \divide\c@pgf@countb by65536\relax%
      \divide\pgf@x by\c@pgf@countb%
      \divide\pgf@y by\c@pgf@countb%
      \pgf@xc=\pgf@x%
      \pgf@yc=8192pt%
      \pgf@y=.125\pgf@y%
      \c@pgf@countb=\pgf@y%
      \divide\pgf@yc by\c@pgf@countb%
      \pgf@process{#1}%
      \pgf@y=\pgf@sys@tonumber{\pgf@yc}\pgf@y%
      \pgf@y=\pgf@sys@tonumber{\pgf@xc}\pgf@y%
      \pgf@process{\pgfpointnormalised{}}%
      \pgf@x=\pgf@sys@tonumber{\pgf@xa}\pgf@x%
      \pgf@y=\pgf@sys@tonumber{\pgf@ya}\pgf@y%
    \else%
      % Ok, now let's compute y/x:
      \c@pgf@countb=\pgf@xa%
      \divide\c@pgf@countb by65536\relax%
      \divide\pgf@x by\c@pgf@countb%
      \divide\pgf@y by\c@pgf@countb%
      \pgf@yc=\pgf@y%
      \pgf@xc=8192pt%
      \pgf@x=.125\pgf@x%
      \c@pgf@countb=\pgf@x%
      \divide\pgf@xc by\c@pgf@countb%
      \pgf@process{#1}%
      \pgf@x=\pgf@sys@tonumber{\pgf@yc}\pgf@x%
      \pgf@x=\pgf@sys@tonumber{\pgf@xc}\pgf@x%
      \pgf@process{\pgfpointnormalised{}}%
      \pgf@x=\pgf@sys@tonumber{\pgf@xa}\pgf@x%
      \pgf@y=\pgf@sys@tonumber{\pgf@ya}\pgf@y%
    \fi%  
  \fi%
}





% Extract the x-coordinate of a point to a dimensions
%  
% #1 = a TeX dimension
% #2 = a point
%
% Example:
%
% \newdimen\mydim
% \pgfextractx{\mydim}{\pgfpoint{2cm}{4pt}}
% % \mydim is now 2cm

\def\pgfextractx#1#2{%
  \pgf@process{#2}%
  #1=1.00374\pgf@x\relax}


% Extract the y-coordinate of a point to a dimensions
%  
% #1 = a TeX dimension
% #2 = a point
%
% Example:
%
% \newdimen\mydim
% \pgfextracty{\mydim}{\pgfpoint{2cm}{4pt}}
% % \mydim is now 4pt

\def\pgfextracty#1#2{%
  \pgf@process{#2}%
  #1=1.00374\pgf@y\relax}


\endinput
